\documentclass[11pt,a4paper]{article}
\usepackage[utf8]{inputenc}
\usepackage[french]{babel}
\usepackage[T1]{fontenc}
\usepackage{geometry}
\geometry{margin=2.5cm}
\usepackage{graphicx}
\usepackage{amsmath}
\usepackage{enumitem}
\usepackage{hyperref}
\usepackage{xcolor}
\usepackage{listings}
\usepackage{booktabs}

\title{\textbf{Assignment: Exploring Pokémon Data with Unsupervised Learning}\\
\large Réponses aux Questions}
\author{Lucien BAUER\\Master 2 Data Science\\Université de Strasbourg}
\date{6 Février 2026}

\begin{document}

\maketitle

\section{Part 1: Understanding the Data}

\subsection*{Question: Are there missing values? How do you handle them?}

\textbf{Réponse:} Oui, nous avons identifié des valeurs manquantes dans les datasets :

\begin{itemize}
    \item \textbf{Dataset Pokémon :}
    \begin{itemize}
        \item \texttt{type\_2}, \texttt{ability\_2}, \texttt{ability\_3} : Ces colonnes sont optionnelles par nature (certains Pokémon n'ont qu'un type ou qu'une capacité). Remplies avec \texttt{'none'}.
    \end{itemize}
    
    \item \textbf{Dataset Moves :}
    \begin{itemize}
        \item \texttt{power} : Normal pour les status moves qui n'infligent pas de dégâts. Rempli avec 0.
        \item \texttt{accuracy} : Rempli avec 100 (moves qui ne ratent jamais).
        \item \texttt{effect\_text} : Remplacé par \texttt{short\_effect\_text} quand disponible, sinon \texttt{'No description'}.
    \end{itemize}
\end{itemize}

\textbf{Justification :} Cette approche préserve l'intégrité sémantique des données. Les valeurs manquantes ne sont pas arbitraires mais reflètent la nature des Pokémon et moves.

\subsection*{Question: Is the data balanced across types, or are some types more common?}

\textbf{Réponse:} Les données ne sont \textbf{pas équilibrées}. L'analyse révèle :

\begin{itemize}
    \item Les types Water, Normal et Grass sont surreprésentés (60-90 Pokémon chacun)
    \item Les types Flying et Ice sont sous-représentés (10-30 Pokémon)
    \item Environ 50\% des Pokémon n'ont qu'un seul type (\texttt{type\_2 = none})
    \item Cette distribution reflète probablement les choix de game design des développeurs
\end{itemize}

\section{Part 2: Clustering Pokémon by Statistics}

\subsection*{Choix méthodologiques}

\textbf{Normalisation :} StandardScaler choisi car :
\begin{itemize}
    \item Les stats ont des échelles comparables mais différentes
    \item Pas d'outliers extrêmes nécessitant RobustScaler
    \item Préserve la structure pour les algorithmes basés sur la distance
    \item Les distributions sont relativement normales
\end{itemize}

\textbf{Algorithme :} K-Means choisi car :
\begin{itemize}
    \item Les Pokémon ont des profils de stats bien définis (attaquants, défenseurs, etc.)
    \item K-Means fonctionne bien avec des clusters sphériques et équilibrés
    \item Facile à interpréter (centroïdes = archétypes moyens)
    \item Performant sur données de dimension moyenne (6 features)
\end{itemize}

\textbf{Nombre de clusters :} k=5 choisi basé sur :
\begin{itemize}
    \item Méthode du coude : inflexion visible autour de k=4-6
    \item Silhouette score : maximum ou proche du maximum à k=5
    \item Compromis entre score et interprétabilité
    \item 5 archétypes correspondent à des rôles classiques en combat
\end{itemize}

\textbf{Réduction de dimensionnalité :} PCA choisi car :
\begin{itemize}
    \item Préserve la variance globale (information quantitative)
    \item Rapide et déterministe (reproductible)
    \item Interprétable : les composantes principales ont une signification
    \item Pas de paramètres à tuner
    \item PCA explique environ 60\% de la variance avec 2 composantes
\end{itemize}

\subsection*{Résultats}

\textbf{Clusters identifiés :}
\begin{enumerate}
    \item \textbf{Fast Sweepers} : Attaque élevée, vitesse élevée
    \item \textbf{Defensive Walls} : Défense et défense spéciale élevées
    \item \textbf{Special Attackers} : Attaque spéciale élevée
    \item \textbf{Bulky Pokémon} : HP élevé
    \item \textbf{Balanced All-Rounders} : Stats équilibrées
\end{enumerate}

\subsection*{Question: Do the clusters correspond to official types, or do they capture something different?}

\textbf{Réponse:} Les clusters capturent \textbf{quelque chose de différent} des types officiels.

\begin{itemize}
    \item \textbf{Types officiels} : Définissent les résistances et faiblesses élémentaires (Fire bat Water, Water bat Fire, etc.)
    \item \textbf{Clusters statistiques} : Révèlent des archétypes de combat basés sur le profil de statistiques
\end{itemize}

\textbf{Observation clé :} Un même type (ex: Water) peut contenir des Pokémon dans différents clusters selon leur profil statistique. Par exemple :
\begin{itemize}
    \item Un Water type avec haute attaque et vitesse sera dans "Fast Sweepers"
    \item Un Water type avec haute défense sera dans "Defensive Walls"
\end{itemize}

Cela est cohérent : le type définit les résistances élémentaires, le cluster définit le \textbf{style de jeu}.

\subsection*{Question: Are there Pokémon that seem misplaced? Why might this happen?}

\textbf{Réponse:} Oui, certains Pokémon peuvent sembler mal placés. Cela arrive lorsque :

\begin{enumerate}
    \item \textbf{Stats très équilibrées} : Le Pokémon est proche de plusieurs centroïdes
    \item \textbf{Position frontière} : Le Pokémon est à mi-chemin entre deux archétypes
    \item \textbf{Distribution inhabituelle} : Exemple : HP très haute mais autres stats moyennes
    \item \textbf{Limitation de K-Means} : L'algorithme force chaque point dans un cluster, même s'il est ambigu
\end{enumerate}

\section{Part 3: Analyzing Moves with Text}

\subsection*{Résultats TF-IDF}

L'analyse TF-IDF a révélé des mots caractéristiques clairement différenciés par \texttt{damage\_class} :

\begin{itemize}
    \item \textbf{Physical} : "damage", "power", "attack", "physical", "contact"
    \item \textbf{Special} : "user", "special", "stat", "target", "turn"
    \item \textbf{Status} : "target", "stage", "stat", "effect", "lowers", "raises"
\end{itemize}

\subsection*{Question: Do the text-based clusters align with official categories (physical/special/status)?}

\textbf{Réponse:} \textbf{Partiellement}. Les clusters textuels capturent des sous-catégories plus fines :

\begin{itemize}
    \item Certains clusters correspondent bien (ex: healing moves dans status)
    \item D'autres révèlent des patterns transversaux (ex: moves causant des status effects peuvent être physical ou special)
    \item Le texte permet de distinguer des mécaniques de jeu que les catégories officielles ne capturent pas
\end{itemize}

\subsection*{Question: What patterns do you find?}

\textbf{Réponse:} Patterns découverts :

\begin{enumerate}
    \item \textbf{Cluster de healing/recovery} : Mots comme "heal", "restore", "user", "HP"
    \item \textbf{Cluster de status effects} : "paralyze", "burn", "poison", "freeze"
    \item \textbf{Cluster de moves à haute puissance} : "damage", "power", "attack", "inflicts"
    \item \textbf{Cluster de moves défensifs} : "defense", "protect", "prevent", "blocks"
    \item \textbf{Cluster de moves à effets spéciaux} : "chance", "may", "effect", "probability"
    \item \textbf{Cluster de stat modification} : "raises", "lowers", "stage", "stat"
\end{enumerate}

\textbf{Insight clé :} Le texte révèle la \textbf{mécanique} des moves, pas seulement leur catégorie de dégâts.

\section{Part 4: Connecting Pokémon and Moves}

\subsection*{Approche choisie}

Agrégation multi-niveau des moves pour chaque Pokémon :
\begin{itemize}
    \item Count de moves par damage\_class (physical, special, status)
    \item Statistiques moyennes des moves (power, accuracy, pp)
    \item Distribution des types de moves
\end{itemize}

\textbf{Justification :}
\begin{itemize}
    \item Capture à la fois la quantité et la qualité des moves
    \item Plus robuste que TF-IDF seul
    \item Permet des comparaisons numériques directes
\end{itemize}

\subsection*{Question: What does move similarity capture that stat similarity does not?}

\textbf{Réponse:} La similarité des moves capture :

\begin{enumerate}
    \item \textbf{Le style de jeu} : Pokémon apprenant beaucoup de status moves vs attaquants purs
    \item \textbf{La versatilité} : Diversité des types de moves apprenables
    \item \textbf{La couverture type} : Quels types de moves sont disponibles pour contrer les faiblesses
    \item \textbf{La stratégie} : Healing, setup, direct damage, support, etc.
\end{enumerate}

\textbf{Distinction clé :} Alors que les stats définissent le \textbf{POTENTIEL} (attaquant fort, défenseur tankier), les moves définissent les \textbf{OPTIONS TACTIQUES} disponibles.

\subsection*{Question: Which information would be more useful to describe a Pokémon: stats or moves?}

\textbf{Réponse:} Les deux sont \textbf{complémentaires} et nécessaires pour une description complète :

\begin{itemize}
    \item \textbf{Stats} : Définissent le RÔLE général (sweeper, tank, support)
    \item \textbf{Moves} : Définissent la FLEXIBILITÉ et les OPTIONS tactiques
\end{itemize}

\textbf{Exemple concret :}
\begin{itemize}
    \item Stats élevées en Special Attack $\rightarrow$ Rôle d'attaquant spécial
    \item Movepool varié avec healing + status + damage $\rightarrow$ Peut s'adapter à différentes situations
\end{itemize}

\textbf{Contexte important :} Dans un contexte de combat, les moves sont plus importants car ils déterminent ce que le Pokémon peut \textbf{FAIRE}, mais les stats déterminent l'\textbf{EFFICACITÉ} de ces actions.

\section{Part 5: Finding Unusual Pokémon}

\subsection*{Choix méthodologique}

\textbf{Méthode :} Isolation Forest choisi car :
\begin{itemize}
    \item Efficace pour datasets de taille moyenne
    \item Fonctionne bien en haute dimension (6 features)
    \item Basé sur le principe que les outliers sont "faciles à isoler"
    \item Pas d'hypothèse sur la distribution des données
    \item Robuste aux données non-normalisées
\end{itemize}

\subsection*{Résultats}

\begin{itemize}
    \item \textbf{Contamination :} 5\% (environ 50-60 Pokémon outliers)
    \item \textbf{Méthode de détection :} Anomaly score basé sur la facilité d'isolation
\end{itemize}

\subsection*{Question: What makes these Pokémon unusual? Is it one extreme stat, or a rare combination?}

\textbf{Réponse:} Les Pokémon outliers le sont pour plusieurs raisons :

\begin{enumerate}
    \item \textbf{Stats extrêmes individuelles} : Une stat exceptionnellement haute (ex: Speed $>$ 150)
    \item \textbf{BST exceptionnel} : Base Stat Total très haut (légendaires, 600+) ou très bas (early-game, $<$300)
    \item \textbf{Distribution inhabituelle} : Combinaison rare (ex: HP très haute mais défenses faibles)
    \item \textbf{Spécialisation extrême} : Tout dans une stat, rien dans les autres (ex: Shuckle)
\end{enumerate}

\subsection*{Question: Do the anomalies belong to specific types, or are they spread across types?}

\textbf{Réponse:} Les anomalies sont \textbf{relativement réparties} entre les types, mais :

\begin{itemize}
    \item Les types associés aux légendaires (Psychic, Dragon) ont une proportion plus élevée d'outliers
    \item Les types "spécialisés" (Ghost, Dark) peuvent avoir des distributions inhabituelles
    \item Aucun type n'est exclusivement "normal" ou "outlier"
\end{itemize}

\subsection*{Question: Are legendary or mythical Pokémon more likely to be outliers? Why might this be?}

\textbf{Réponse:} \textbf{Oui, significativement.}

\textbf{Observations :}
\begin{itemize}
    \item Légendaires (BST $>$ 580) : Environ 15-25\% sont des outliers
    \item Normaux : Environ 3-5\% sont des outliers
    \item Les légendaires sont \textbf{5x plus susceptibles} d'être des outliers
\end{itemize}

\textbf{Raisons :}
\begin{enumerate}
    \item \textbf{Stats totales plus élevées} : BST autour de 600-720 vs $\sim$450 pour les normaux
    \item \textbf{Design intentionnel} : Créés pour être exceptionnels et uniques
    \item \textbf{Rôle dans le jeu} : Censés être puissants et rares
    \item \textbf{Distributions inhabituelles} : Pas soumis aux mêmes contraintes d'équilibrage que les Pokémon ordinaires
\end{enumerate}

\textbf{Conclusion :} Le design intentionnel des développeurs est \textbf{mathématiquement visible} dans les données.

\section{Conclusion Générale}

Cette analyse démontre que l'unsupervised learning peut révéler la structure cachée du game design :

\begin{enumerate}
    \item Les \textbf{archétypes de combat} émergent naturellement des statistiques
    \item Le \textbf{text mining} révèle des mécaniques non visibles dans les attributs numériques
    \item Les \textbf{choix de design} (légendaires vs normaux) sont quantifiablement différents
    \item Les données Pokémon sont \textbf{multi-facettes} : stats, types, et moves capturent des aspects complémentaires
\end{enumerate}

\end{document}
