\documentclass[11pt]{beamer}
\usetheme{Madrid}
\usecolortheme{default}

\usepackage[utf8]{inputenc}
\usepackage[french]{babel}
\usepackage[T1]{fontenc}
\usepackage{graphicx}
\usepackage{booktabs}

\title{Exploring Pokémon Data with Unsupervised Learning}
\author{Lucien BAUER}
\institute{Master 2 Data Science\\Université de Strasbourg}
\date{6 Février 2026}

\begin{document}

\frame{\titlepage}

% ============================================================================
\begin{frame}{Vue d'ensemble}
\begin{block}{Objectif}
Découvrir des patterns cachés dans les données Pokémon en utilisant l'unsupervised learning
\end{block}

\begin{block}{Datasets}
\begin{itemize}
    \item \textbf{Pokémon} : $\sim$1000 Pokémon avec stats (HP, Attack, Defense, etc.)
    \item \textbf{Moves} : $\sim$900 moves avec descriptions textuelles
    \item \textbf{Learnset} : Qui apprend quoi ? ($\sim$50 moves/Pokémon en moyenne)
\end{itemize}
\end{block}

\begin{block}{Question principale}
Les Pokémon se regroupent-ils naturellement selon leurs caractéristiques ?
\end{block}
\end{frame}

% ============================================================================
\begin{frame}{Part 1: Understanding the Data}

\begin{columns}[T]
\column{0.5\textwidth}
\textbf{Statistiques clés}
\begin{itemize}
    \item 1000+ Pokémon
    \item 900+ moves
    \item $\sim$50 moves/Pokémon
    \item 18 types différents
\end{itemize}

\vspace{0.5cm}
\textbf{Valeurs manquantes}
\begin{itemize}
    \item \texttt{type\_2} : optionnel
    \item \texttt{power} : 0 pour status
    \item Approche sémantique
\end{itemize}

\column{0.5\textwidth}
% INSTRUCTION: Insérer ici le graphique "Distribution des Types Primaires"
% depuis le notebook (Part 1, cellule distribution)
\begin{center}
\textit{[Graphique: Distribution Types]}
\end{center}

\vspace{0.3cm}
\textbf{Observation}
\begin{itemize}
    \item Données déséquilibrées
    \item Water, Normal, Grass surreprésentés
\end{itemize}
\end{columns}

\end{frame}

% ============================================================================
\begin{frame}{Part 2: Clustering - Méthodologie}

\begin{block}{Choix techniques}
\begin{itemize}
    \item \textbf{Normalisation} : StandardScaler (préserve structure)
    \item \textbf{Algorithme} : K-Means (archétypes bien définis)
    \item \textbf{k=5 clusters} : Elbow + Silhouette ($\sim$0.35-0.40)
    \item \textbf{Visualisation} : PCA (60\% variance expliquée)
\end{itemize}
\end{block}

\begin{center}
% INSTRUCTION: Insérer ici les graphiques Elbow + Silhouette
% depuis le notebook (Part 2, cellule Elbow Method)
\textit{[Graphiques: Elbow Method + Silhouette Score]}
\end{center}

\end{frame}

% ============================================================================
\begin{frame}{Part 2: Les 5 Archétypes}

\begin{columns}[T]
\column{0.5\textwidth}
% INSTRUCTION: Insérer ici le scatter plot PCA coloré par clusters
% depuis le notebook (Part 2, cellule Visualisation PCA)
\begin{center}
\textit{[Graphique: PCA par Clusters]}
\end{center}

\column{0.5\textwidth}
\textbf{Clusters identifiés}
\begin{enumerate}
    \item \textcolor{blue}{Fast Sweepers}\\
    {\small Att. élevée + vitesse}
    
    \item \textcolor{red}{Defensive Walls}\\
    {\small Def. + Def. Spé élevées}
    
    \item \textcolor{green}{Special Attackers}\\
    {\small Att. Spé élevée}
    
    \item \textcolor{orange}{Bulky Pokémon}\\
    {\small HP élevé}
    
    \item \textcolor{purple}{Balanced}\\
    {\small Stats équilibrées}
\end{enumerate}
\end{columns}

\end{frame}

% ============================================================================
\begin{frame}{Part 2: Insight Principal}

\begin{alertblock}{Découverte clé}
Les clusters NE correspondent PAS aux types officiels !
\end{alertblock}

\begin{columns}[T]
\column{0.5\textwidth}
% INSTRUCTION: Insérer ici la heatmap des stats moyennes par cluster
% depuis le notebook (Part 2, cellule heatmap)
\begin{center}
\textit{[Heatmap: Stats par Cluster]}
\end{center}

\column{0.5\textwidth}
\textbf{Interprétation}
\begin{itemize}
    \item \textbf{Types} = Résistances élémentaires
    \item \textbf{Clusters} = Rôles de combat
\end{itemize}

\vspace{0.3cm}
\textbf{Exemple}
\begin{itemize}
    \item Un Pokémon Water peut être dans \textit{n'importe quel} cluster selon ses stats
    \item Le type définit "contre quoi il est fort"
    \item Le cluster définit "comment il joue"
\end{itemize}
\end{columns}

\end{frame}

% ============================================================================
\begin{frame}{Part 3: Text Analysis}

\begin{block}{TF-IDF sur descriptions des moves}
Révèle des sous-catégories fines au-delà de physical/special/status
\end{block}

\begin{center}
% INSTRUCTION: Insérer ici le tableau des mots caractéristiques
% ou un graphique comparatif
\begin{tabular}{lll}
\toprule
\textbf{Physical} & \textbf{Special} & \textbf{Status} \\
\midrule
damage & user & target \\
power & special & stage \\
attack & stat & effect \\
contact & turn & lowers \\
physical & target & raises \\
\bottomrule
\end{tabular}
\end{center}

\vspace{0.3cm}
\textbf{Patterns découverts}
\begin{itemize}
    \item Healing moves, Status effects, High-damage, Defensive, Stat modification
    \item Le texte capture la \textbf{mécanique}, pas juste la catégorie
\end{itemize}

\end{frame}

% ============================================================================
\begin{frame}{Part 4: Stats vs Moves}

\begin{block}{Question}
Les Pokémon similaires en stats sont-ils similaires en moves ?
\end{block}

\textbf{Approche}
\begin{itemize}
    \item Représentation move-based : counts + stats moyennes + types
    \item Comparaison de similarités
\end{itemize}

\begin{columns}[T]
\column{0.5\textwidth}
\textbf{Découverte}
\begin{itemize}
    \item Corrélation \textbf{modérée}
    \item Stats = POTENTIEL
    \item Moves = OPTIONS
    \item Les deux sont complémentaires
\end{itemize}

\column{0.5\textwidth}
% INSTRUCTION: Insérer ici le scatter plot corrélation similarités
% ou le tableau de comparaison
\begin{center}
\textit{[Graphique: Corrélation\\Stats vs Moves]}
\end{center}
\end{columns}

\end{frame}

% ============================================================================
\begin{frame}{Part 5: Anomaly Detection}

\begin{block}{Méthode}
Isolation Forest (contamination = 5\%)
\end{block}

\begin{columns}[T]
\column{0.5\textwidth}
% INSTRUCTION: Insérer ici le scatter plot PCA avec outliers en rouge
% depuis le notebook (Part 5, cellule visualisation outliers)
\begin{center}
\textit{[Graphique: Outliers\\dans l'espace PCA]}
\end{center}

\column{0.5\textwidth}
\textbf{Résultats}
\begin{itemize}
    \item $\sim$5\% outliers (50-60 Pokémon)
    \item Stats extrêmes ou distributions inhabituelles
\end{itemize}

\vspace{0.3cm}
\textbf{Causes}
\begin{enumerate}
    \item Une stat extrême
    \item BST exceptionnel
    \item Combinaison rare
    \item Spécialisation extrême
\end{enumerate}
\end{columns}

\end{frame}

% ============================================================================
\begin{frame}{Part 5: Légendaires vs Normaux}

\begin{alertblock}{Découverte majeure}
Les légendaires sont 5x plus susceptibles d'être des outliers !
\end{alertblock}

\begin{center}
\begin{tabular}{lcc}
\toprule
& \textbf{Légendaires} & \textbf{Normaux} \\
& (BST $>$ 580) & \\
\midrule
Total & $\sim$100 & $\sim$900 \\
Outliers & 15-25\% & 3-5\% \\
\bottomrule
\end{tabular}
\end{center}

\vspace{0.5cm}
\textbf{Interprétation}
\begin{itemize}
    \item Design intentionnel : légendaires \textit{doivent} être exceptionnels
    \item Les choix des développeurs sont \textbf{mathématiquement visibles}
    \item Pas soumis aux mêmes contraintes d'équilibrage
\end{itemize}

\end{frame}

% ============================================================================
\begin{frame}{Conclusion}

\begin{block}{Découvertes principales}
\begin{enumerate}
    \item \textbf{5 archétypes} émergent des stats (Fast Sweepers, Walls, etc.)
    \item Les clusters révèlent des \textbf{rôles de combat}, pas des types
    \item Le \textbf{text mining} capture des mécaniques au-delà des catégories
    \item Stats et moves sont \textbf{complémentaires} (potentiel vs options)
    \item Les légendaires sont \textbf{mathématiquement différents} par design
\end{enumerate}
\end{block}

\begin{block}{Insight général}
L'unsupervised learning révèle la \textbf{structure cachée du game design}
\end{block}

\end{frame}

% ============================================================================
\begin{frame}{Limites et Extensions}

\textbf{Limites}
\begin{itemize}
    \item K-Means force chaque Pokémon dans un cluster (pas de flou)
    \item PCA perd 40\% de variance
    \item Text analysis limité par qualité des descriptions
\end{itemize}

\vspace{0.5cm}
\textbf{Extensions possibles}
\begin{itemize}
    \item Analyse temporelle : évolution des designs à travers les générations
    \item Network analysis : graphes de similarité
    \item Supervised learning : prédire le tier compétitif
    \item Recommender system : suggérer des équipes équilibrées
\end{itemize}

\end{frame}

% ============================================================================
\begin{frame}[standout]
\Huge Questions ?

\vspace{1cm}
\normalsize
Merci pour votre attention !
\end{frame}

\end{document}
